\documentclass[a4paper,twoside,11pt]{article}

\usepackage[utf8]{inputenc}

\setlength{\textwidth}{150mm}
\setlength{\textheight}{230mm}
\setlength{\oddsidemargin}{5mm}
\setlength{\evensidemargin}{5mm}
\setlength{\topmargin}{-10mm}

\title{Dependencias funcionales}
\author{Hiram Ehecatl Lujano Pastrana (313095409)}

\begin{document}
\maketitle

\textbf{Geografico}:
\begin{itemize}
  \item Estado: Solo tenemos al id\_estado que determina al nombre\_estado
        y abreviatura. Formalmente tenemos

        Estado(id\_estado, nombre\_estado, abreviatura)\\
        $F=\{id\_estado \rightarrow nombre\_estado~abreviatura\}$

        Como id\_estado es llave y esta a la izquierda de la unica DF implica
        que la relación esta en 3FN.

  \item Municipio: Solo tiene a id\_estado e id\_municipio que determinan
        al nombre\_municipio.
        Formalmente

        Municipio(id\_estado, id\_municipio, nombre\_municipio)\\
        $F=\{id\_estado~id\_municipio\rightarrow nombre\_municipio\}$

        (id\_estado~id\_municipio) es llave y esta es parte izquierda
        de la unica DF por lo que la relación esta en 3FN.

  \item Distrito\_local: id\_estado e id\_distrito\_local determinan al
        nombre\_distrito\_local. Formalmente

        Distrito\_local(id\_estado, id\_distrito\_local, nombre\_distrito\_local)\\
        $F=\{id\_estado~id\_distrito\_local\rightarrow nombre\_distrito\_local\}$

        (id\_estado~id\_distrito\_local) es llave y esta del lado izquierdo de
        la unica DF por lo que la relación esta en 3FN.

  \item Distrito\_federal: id\_estado e id\_distrito\_federal determinan al
        nombre\_distrito\_federal. Formalmente

        Distrito\_federal(id\_estado, id\_distrito\_federal, nombre\_distrito\_federal)\\
        $F=\{id\_estado~id\_distrito\_federal\rightarrow nombre\_distrito\_federal\}$

        (id\_estado~id\_distrito\_federal) es llave y esta del lado izquierdo de
        la unica DF por lo que la relación esta en 3FN.

  \item Seccion: No tiene DF no triviales

\end{itemize}

\textbf{Casillas}:
\begin{itemize}
  \item Casilla: id\_estado, id\_municipio, id\_distrito\_local, id\_distrito\_federal, seccion
        e id\_casilla determinan a tipo\_casilla y aprobada. Formalmente

        Casilla(id\_estado, id\_municipio, id\_distrito\_local, id\_distrito\_federal,
         seccion, id\_casilla, tipo\_casilla, aprobada)\\
        $F=\{id\_estado~id\_municipio~id\_distrito\_local~id\_distrito\_federal~seccion~id\_casilla\rightarrow\\
        tipo\_casilla~aprobada\}$

        (id\_estado~id\_municipio~id\_distrito\_local~id\_distrito\_federal~seccion~id\_casilla) es
        llave y esta del lado izquierdo de la unica DF, por lo que la relación esta en 3FN.
\end{itemize}

\textbf{Partidos\_politicos}:
\begin{itemize}
  \item Partido: id\_distrito\_federal, id\_partido determinan el nombre\_partido y las siglas.
        Formalmente

        Partido(id\_distrito\_federal, id\_partido, nombre\_partido, siglas)\\
        $F=\{id\_distrito\_federal~id\_partido\rightarrow nombre\_partido~siglas\}$

        (id\_distrito\_federal~id\_partido) es llave y esta del lado izquierdo de la unica DF,
        por lo que la relación esta en 3FN.
\end{itemize}

\textbf{Representantes}:
\begin{itemize}
  \item Representante\_preliminar: id\_distrito\_federal, id\_partido, id\_representante determinan
        a nombre\_representante y fecha\_y\_hora\_registro. Formalmente

        Representante\_preliminar(id\_distrito\_federal, id\_partido, id\_representante,
        nombre\_representante, fecha\_y\_hora\_registro)\\
        $F=\{id\_distrito\_federal~id\_partido~id\_representante\rightarrow\\
        nombre\_representante~fecha\_y\_hora\_registro\}$

        (id\_distrito\_federal~id\_partido~id\_representante) es llave y esta del lado izquierdo de la unica DF,
        por lo que la relación esta en 3FN.

  \item Representante\_aprobado: id\_estado, id\_distrito\_federal, id\_representante determinan
        id\_partido\_que\_registro, fecha\_y\_hora\_aprobacion y usuario\_que\_aprobo. Formalmente

        Representante\_aprobado(id\_estado, id\_distrito\_federal, id\_representante, id\_partido\_que\_registro
        fecha\_y\_hora\_aprobacion, usuario\_que\_aprobo)\\
        $F=\{id\_estado~id\_distrito\_federal~id\_representante\rightarrow\\
        id\_partido\_que\_registro~fecha\_y\_hora\_aprobacion~usuario\_que\_aprobo\}$

        (id\_estado~id\_distrito\_federal~id\_representante) es llave y esta del lado izquierdo de la unica DF,
        por lo que la relación esta en 3FN.

  \item Representante\_general: id\_estado, id\_distrito\_federal, id\_representante determinan
        direccion\_representante\_g y clave\_elector. Formalmente

        Representante\_general(id\_estado, id\_distrito\_federal, id\_representante,
        direccion\_representante\_g, clave\_elector)\\
        $F=\{id\_estado~id\_distrito\_federal~id\_representante\rightarrow\\
        direccion\_representante\_g~clave\_elector\}$

        (id\_estado~id\_distrito\_federal~id\_representante) es llave y esta del lado izquierdo de la unica DF,
        por lo que la relación esta en 3FN.

  \item Representante\_ante\_casilla: id\_estado, id\_distrito\_federal, id\_representante determinan
        id\_casilla, direccion\_representante\_ac y tipo\_cargo. Formalmente

        Representante\_ante\_casilla(id\_estado, id\_distrito\_federal, id\_representante,
        id\_casilla, direccion\_representante\_ac, tipo\_cargo)\\
        $F=\{id\_estado~id\_distrito\_federal~id\_representante\rightarrow\\
        id\_casilla~direccion\_representante\_ac~tipo\_cargo\}$

        (id\_estado~id\_distrito\_federal~id\_representante) es llave y esta del lado izquierdo de la unica DF,
        por lo que la relación esta en 3FN.

  \item Asistencia: id\_representante y fecha\_y\_hora determinan a tipo\_presencia, registro\_presencia. Formalmente

        Asistencia(id\_representante, fecha\_y\_hora,tipo\_presencia, registro\_presencia)\\
        $F=\{id\_representante, fecha\_y\_hora \rightarrow tipo\_presencia, registro\_presencia\}$

        (id\_representante, fecha\_y\_hora) es llave y esta del lado izquierdo de la unica DF,
        por lo que la relación esta en 3FN.
        
  \item Domicilia: Aqui solo hay DF triviales.
  \item Casilla\_que\_representa: Solo DF triviales.
  \item log\_representantes\_aprobados: Sin DF.
  \item Representantes\_sustituciones: Solo DF triviales.

\end{itemize}

\end{document}
