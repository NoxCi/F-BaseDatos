\documentclass[a4paper,twoside,11pt]{article}

\usepackage[utf8]{inputenc}

\setlength{\textwidth}{125mm}
\setlength{\textheight}{195mm}
\setlength{\topmargin}{-5mm}

\title{Dependencias funcionales}
\author{Hiram Ehecatl Lujano Pastrana (313095409)}

\begin{document}
\maketitle

Tomando es cuenta que cada entidad tiene su propia id unica que
determina a sus demas atributos, podemos ahorrarnos mucha escritura
diciendo que si la entidad A determina a la entidad B signica que el
id de A determina el id de B y por lo tanto a sus atributos.

Empezando con el esquema Geografico, podemos ver gracias al esquema
relacional y al diagrama E-R que Municipio, Distrito\_local y
Distrito\_federal determinan al Estado y que Seccion determina a
Municipio, Distrito\_local, Distrito\_federal y al Estado.

Con Casillas, tenemos que una Casilla determina a la Seccion que la
contiene.

El esquema de Partidos Politicos no tiene otra dependencia mas que
su propia entidad Partido.

Pasando al esquema de Representantes tenemos que la entidad
Representante\_preliminar determina al Partido al que pertenece,
Representante\_aprobado determina al Representante\_preliminar que fue
aprobo, asi como al Representante\_general y ante casilla pues no pueden
ser representantes generales o ante casilla si no fueron aprobados.

\end{document}
